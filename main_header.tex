\usepackage[ngerman]{babel}
\usepackage[T1]{fontenc}
\usepackage[utf8x]{inputenc}
\usepackage{color}
\usepackage{colortbl}
\usepackage{listings}
\usepackage{longtable}
\usepackage{tikz}
\usepackage{amsthm}
\usepackage{amsmath}
\usepackage{pgfplots}
\usetikzlibrary{shapes, calc}
\usetheme[
  outer/progressbar=foot,
  outer/numbering=none
]{metropolis}

\def\datengerman{\def\today{\ifnum\day<10 0\fi\number\day.\ifnum\month<10 0\fi\number\month.\number\year}}
\date{\today}

\newcommand{\befehl}[1]{\ttfamily #1\sf}

\DeclareMathOperator{\e}{e}
\newcommand{\iu}{{\mathrm{i}\mkern1mu}}

\definecolor{SimpsonsYellow}{RGB}{255,217,15}

\setbeamercolor{progress bar in head/foot}{fg=SimpsonsYellow}
\setbeamercolor{title separator}{fg=SimpsonsYellow}
\setbeamercolor{frametitle}{fg=black,bg=SimpsonsYellow}

\title{The Simpsons meet IT}
\subtitle{Was wir von den Simpsons und Futurama über Mathematik, Kryptografie, IT und Quantenphysik lernen können}


\only<article>{
  \publishers{\textcopyright\ Markus Flingelli, 2016 - \the\year}
}

\only<article>{
  \hypersetup{
  pdftitle = {The Simpsons meet IT -- Was wir von den Simpsons und Futurama über Mathematik, Kryptografie, IT und Quantenphysik lernen können},
  pdfsubject = {},
  pdfkeywords = {Simpsons, IT, Mathematik, Informatik, Physik},
  pdfauthor = {\textcopyright\ Markus Flingelli, 2016 - \the\year},
  pdfcreator = {\LaTeX\ u.a. mit dem Paket \flqq hyperref\frqq},
  pdfproducer = {pdfeTeX-0.\the\pdftexversion\pdftexrevision},
  }
}

\hypersetup{colorlinks=true, linkcolor=black, urlcolor=blue}

\definecolor{Titelfarbe}{RGB}{245,255,250}
\definecolor{Titelhintergrundfarbe}{RGB}{255,217,15}
\definecolor{ivory}{RGB}{255,255,240}
\definecolor{basicGreen}{RGB}{0,140,0}
\definecolor{basicRed}{RGB}{128,0,0}

\theoremstyle{plain}
\newtheorem{lem}{Lemma}

\lstdefinestyle{base}{
  language={[Visual]Basic},
  emptylines=1,
  breaklines=true,
  basicstyle=\ttfamily\color{basicRed},
  moredelim=**[is][\color{basicGreen}]{@}{@},
}

\AtBeginSection[]
{
  \begin{frame}<beamer>
    \frametitle{Gliederung}
    \tableofcontents[sections={\thesection}]
  \end{frame}
  \begin{frame}<handout>
    \frametitle{Gliederung}
    \tableofcontents[sections={\thesection}]
  \end{frame}
}