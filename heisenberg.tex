\subsection{Heisenberg}

\frame{
\only<presentation>{
	\frametitle{Heisenbergsche Unschärferelation}
}
\begin{block}{Definition:}
\begin{itemize}
	\item Komplementäre Eigenschaften eines Teilchens sind nicht gleichzeitig beliebig genau bestimmbar
	\item Beispiel: Ort und Impuls
\end{itemize}
	\[\Delta x \cdot \Delta p \geq \frac{h}{2\pi}
\]
\end{block}
\begin{block}{Erläuterung}
Plancksches Wirkungsquantum: $h = 6,626069\cdot 10^{-34}J\cdot s$
\end{block}
}

\only<article>{
\noindent\Hinweis[0.975\textwidth]{t}{Anmerkung}{% 
Besitzt ein Körper der Masse $m$ die Geschwindigkeit $\vec{v}$, so definiert man als Impuls des Körpers den Vektor
\[
	\vec{p} = m\cdot \vec{v}
\]
}
}

\frame{
\only<presentation>{
	\frametitle{Heisenbergsche Unschärferelation}
}
\begin{block}{Das Glück des Phillip J. Fry}
\begin{itemize}
	\item Beim Pferderennen beschwert sich Professor Farnsworth über das Ergebnis eines Quantumzieleinlaufs
	\item Ergebnis sei durch Messung verfälscht worden
	\item Siehe: \href{https://www.youtube.com/watch?v=AuFUnKr8VVI}{\url{https://www.youtube.com/watch?v=AuFUnKr8VVI}}
\end{itemize}
\end{block}
}