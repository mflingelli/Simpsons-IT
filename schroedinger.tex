\subsection{Schrödinger}

\frame{
\only<presentation>{
	\frametitle{Quantenphysik}
}
\begin{block}{Law and Oracle}
\begin{itemize}
	\item Erwin Schrödinger wird von der Polizei aufgehalten, weil er mit fünf Meilen pro Stunde schneller als Lichtgeschwindigkeit unterwegs war
	\item Nachdem ihn die Polizei aufgehalten hat, will er nicht sagen, was sich in der Kiste auf dem Beifahrersitz befindet
	\item Fry öffnet die Kiste und eine Katze springt ihn an
	\item Siehe: \href{https://www.youtube.com/watch?v=zcqIXS1gGdU}{\url{https://www.youtube.com/watch?v=zcqIXS1gGdU}}
\end{itemize}
\end{block}
}

\subsubsection{Schrödingers Katze}

\frame{
\only<presentation>{\frametitle{Schrödingers Katze}}
\begin{block}{}
\begin{itemize}
	\item Gedankenexperiment zur Überlagerung von Zuständen in der Quantenmechanik
	\item Superposition der Zustände
\end{itemize}
\end{block}
}

\only<article>{
\noindent\Hinweis[0.975\textwidth]{t}{Anmerkungen}{% 
\begin{itemize}
  \item Schrödingers Gedankenexperiment wird auch in der Serie \glqq The Big Bang Theory\grqq\ erwähnt (siehe \href{https://www.youtube.com/watch?v=wTJnRS8SZhI}{https://www.youtube.com/watch?v=wTJnRS8SZhI}).
  \item Hier wird dabei der Status der Beziehung zwischen Penny und Leonard thematisiert.
\end{itemize}
}
}

\subsubsection{Schrödingergleichung}

\frame{
\only<presentation>{\frametitle{Schrödingergleichung}}
\begin{block}{Definition:}
Schrödingergleichung in ihrer allgemeinsten Form lautet:
\begin{displaymath}
i\hbar \frac{\partial}{\partial t}\Psi(x,t) = -\frac{\hbar^2}{2m}
\frac{\partial^2 }{\partial x^2}\Psi(x,t) + V(x,t) \Psi(x,t)
\end{displaymath}
\end{block}
\begin{block}{Erläuterungen:}
\begin{itemize}
  \item $\hbar = \frac{h}{2\pi}$
  \item Wellenfunktion: $\Psi(x,t)$
  \item Potential: $V(x,t)$
\end{itemize}
\end{block}
}

\only<article>{
\noindent\Hinweis[0.975\textwidth]{t}{Anmerkungen}{% 
\begin{itemize}
  \item Bei der Schrödinger-Gleichung handelt es sich um eine Gleichung, die eine der klassischen Wellengleichung ähnlichen Wellenfunktion ($\Psi$-Funktion) enthält, mit der sich die Ausbreitung von Teilchen im Raum, oder genauer, die Wahrscheinlichkeit, Teilchen in einem bestimmten Raumvolumen nachzuweisen, beschreiben lässt.
  \item Die Schrödingergleichung kann nicht mathematisch exakt hergeleitet werden. Sie wurde von Schrödinger postuliert und hat sich zur Beschreibung quantenmechanischer Systeme als richtig erwiesen.
  \item Ein Spezialfall der Schrödingergleichung ist die zeitunabhängige Schrödingergleichung, die Systeme beschreibt, bei denen die Funktionswerte $\Psi(x)$ bzw. $\Psi(r)$ und die potentielle Energie $E_{pot}$ nur vom Ort und nicht von der Zeit abhängen. Solche Zustände werden als stationäre Zustände bezeichnet.
\end{itemize}
}
}